
**Note - Ellen will edit and add to this Thursday and add citations, but I wanted to add something for now.

Many statistical agencies provide microdata files directly to the public, available online without requiring users to register or provide information about what they will do with it. 

One advantage of making public use files available is that the data can reach a much wider audience and be used much more broadly. They are generally made free for anyone who would like access, which makes them particularly appealing for students and researchers alike.

Depending on the data, they can also be extremely easy for people to use and understand. While some may require extensive codebooks and statistical knowledge to manipulate, some government agencies are moving toward online query systems that allow anyone to determine information with little effort. For example, Baltimore city has a public database of all 911 calls that can be downloaded if the user would like to. However, they also have a query system on their website that allows users to look at combinations of variables, such as locations, date and times, and types of emergencies. With a few mouse clicks, any user can create a visualization of the data that can include multiple variables. 

Providing more people easier access to microdata can allow for more queries to be run, more data to be combined, and more research to be conducted that can change public policy. Most government agencies conduct marketing research and implement marketing techniques with the goal of having more people use their data, whether tabular or microdata. By allowing access to microdata, however, there is more information that can be gathered from the data, it can be used in additional ways, and has the potential to influence more research, more policies, and more people.

In addition, wide access to public use data can often allow for those datasets to be combined with other confidential datasets that likely wouldn’t be possible if the data were only available in a restricted use setting. Most restricted use access options, such as FSRDCs or virtual data enclaves, prevent users from removing the data in any means from their location. Public use files can usually be uploaded into these settings, but because it is not possible to pull the restricted use files from their location, it is often impossible to combine two restricted use files and draw conclusions from the resulting datasets. Public use files allow data users to combine the data with confidential datasets, which can result in conclusions that would not be possible to draw without public access to the data. 

Of course, the practice of making access to data easier, especially unperturbed microdata, increases the risk of it being used for malfeasance. Even data that are believed to be anonymized can often be used to determine information about specific respondents. Examples of misused public data are numerous, from determining preferences of Netflix users  to identifying the governor of Massachusetts in medical records by pairing it with public voter identification information . The Panel of Data Access for Research Purposes proposed two recommendations to limit de-identification attacks on government data. First, all users should be notified when accessing government data that it was collected with the legal requirement to be used for only statistical purposes and users should be required to acknowledge that they have read the disclaimer before the data can be viewed. Second, government institutions should impose ``meaningful’’ penalties for users who use the data for something other than statistical purposes, such as de-identification attempts. While that recommendation was made by the panel in 2005, many government agencies have not taken these steps and still allow access to their microdata without reading a disclaimer and have no real penalties in place for misuse of the data.

While the United States has decentralized statistical agencies which largely create their own rules on data access, other countries often have tiers of access that are standard for all data programs. Stats Canada, for example, has certain datasets that are available publicly and certain datasets that can only be accessed through data research centers or remote access that is granted for researchers. However, there are also certain datasets for which Stats Canada allows researchers to access tabular data by submitting statistical programs without seeing the underlying microdata, for which tabular data is returned. This intermediate data access level can allow the program to apply disclosure methods to the tabular data but allow researchers to use the microdata, which can add some protections for the respondents.

The United Kingdom Data Center has three levels of access, as well: open, safeguarded, and controlled data. Open data is believed to be completely free of identifying data and can be used by anyone without permission or registration. Safeguarded data is believed not to have identifying information, but there could be concerns about the possibility of identifying respondents by linking the information with other dataset. To gain access to these data, users need to register and agree to certain conditions, such as not using the data to identify individuals or disseminate any identifying information. Finally, controlled data is believed to contain information that could be disclosive and is protected through a number of means, such as requiring users to register and be approved, as well as complete training to access the data.

For both Stats Canada and the UK Data Center, each dataset is evaluated for risk and the microdata are assigned a level of protection. In some cases with risky data, the data are altered in some way, such as by removing variables, to decrease the risk. This can allow different versions of the same dataset to be available at different access levels, so researcher with training and appropriate credentials can gain access to the full data while other users who may not have the necessary credentials or may not need the full dataset can use the modified data. This gives the advantage of wide access to the data while protecting the most vulnerable elements of information from the respondents.

\subsection{Disclosure Protection in Public Use Files}

When datasets are release for public use, they generally have some modifications made to reduce risk to respondents. That does not mean that all risk is eliminated, but most government agencies at least attempt to limit the likelihood of re-identification. At the very least, this means removing respondent identifying information, such as names, addresses, and dates of birth from the data. In many cases, this is not removed from the file entirely, but changed into some sort of aggregate variable like county or state rather than address, or year of birth rather than exact date. While this helps to protect the respondents, it can also reduce the utility of the file. If state is the only information provided regarding the location of the respondent, it prevents studies of rural versus urban respondents.

In addition, many variables in public use microdata are rounded, top-coded, or organized into certain groups. Again, these methods may reduce some amount of utility, but can protect respondents. With the availability of information available that can be merged with public use data, exact information about variables such as dollar amounts, especially for outliers, can pose a risk. However, these methods can also skew; if wages are top-coded so all earnings above a certain level are written as something lower, the averages will no longer accurately reflect the overall population.

More surveys are beginning to use some form of data perturbation to create the microdata file that is released to the public. 
The American Community Survey (ACS) at the Census Bureau, which collects information from 2.3 million housing units per year, will move to creating synthetic data and release that as their public use microdata file, rather than the actual collected data. They have begun testing methods for creating the synthetic data, but so far, only some of the properties in the original data are reflected in the synthetic data. At geographic areas lower than the state, there are even fewer properties of the original data reflected in the synthetic microdata file.

Some surveys, such as the National Longitudinal Survey of Youth at the Bureau of Labor Statistics, release public use data files with most variables include, but lacking almost all geographic information. While that helps to provide protection for respondents to the surveys, it makes it impossible to make determinations about variations across locations.

