\documentclass[12pt]{article}

%%% All setup/preamble moved to a different file
%%% from Anna
\usepackage{ifthen,pifont,latexsym}
\usepackage[T1]{fontenc}
\usepackage{times,avant}
\usepackage{comment}
\usepackage{datetime}

%%\usepackage{mathrsfs}
%\usepackage[mtpluscal,mtplusscr,T1]{mathtime}
\usepackage[round,authoryear]{natbib}
%%
\usepackage[pdftex]{graphicx}
%%
\def\Includefigs{true}
%%\usepackage{amsmath}
%%
%% \usepackage{draftmark}\markdraftpage
%%
%%%%%%%%%% Other preamble commands here ...
%%
%%
\def\rfeqn#1{(\ref{eq.#1})}
\def\vs#1{\vspace{#1\baselineskip}}
\def\HeadSpace{\vs{.17}}
\def\BoldHead#1{\HeadSpace\textbf{#1}}
\def\ItalHead#1{\HeadSpace\textit{#1}}
\def\UlHead#1{\HeadSpace\underline{#1}}
\def\floatpagefraction{.7}
%%
\newenvironment{descz}{\begin{description}\def\itemsep{0in}}%%
{\end{description}}
\newenvironment{itemz}{\begin{itemize}\def\itemsep{0in}}%%
{\end{itemize}}
\newenvironment{enumz}{\begin{enumerate}\def\itemsep{0in}}%%
{\end{enumerate}}
\newcommand{\m}{\ensuremath{\mathbf{\mu}}}
\newcommand{\s}{\ensuremath{\mathbf{\Sigma}}}
\newcommand{\bx}{\textbf{x}}


%%\input acronyms.tex
%%
%% Def's and macros
%%
% \mathscr changed to \mathcal in the following by APS
\def\DB{\ensuremath{\mathbf{D}}} %% Database
\def\QS{\ensuremath{\mathscr{Q}}} %% Query space
\def\DR{\ensuremath{\mathbf{DR}}} %% Disclosure risk
\def\DU{\ensuremath{\mathbf{DU}}} %% Data utility
\def\DD{\ensuremath{\mathbf{DD}}} %% Data distortion
\def\RS{\ensuremath{\mathbf{R}}} %% Release space
\def\DBORIG{\ensuremath{{D}_{\mathrm{orig}}}} %% Database
\def\DBPOST{\ensuremath{\mathscr{D}_{\mathrm{post}}}} %% Database
\def\DBREL{\ensuremath{D_{\mathrm{rel}}}} %% Released database
%% Methods
\def\NOI{{\tt Noise}}
\def\MICZ{{\tt Micz}}
\def\MICP{{\tt Micp}}
\def\MICM{{\tt Micm_all}}
\def\MICI{{\tt Micir}}
\def\MIC7{{\tt Micm_groups}}
\def\RESAMP{{\tt Resamp}}
\def\RANK{{\tt Rank}}
%% Utility measures
\def\LR{\ensuremath{\mathbf{LR}}} %% Likelihood ratio
\def\KL{\ensuremath{\mathbf{KL}}} %% KL measure
\def\IO{\ensuremath{\mathbf{IO}}} %% interval overlap measure
\def\EO{\ensuremath{\mathbf{EO}}} %% ellipsoid overlap measure
%\def\argmax{\operatornamewithlimits{arg\max}}
\def\SDL{statistical disclosure limitation (SDL)%%
    \gdef\SDL{SDL}}
    
%%%%%% END ANNA CONFIG

\usepackage{hyperref}


\begin{comment}
Place \usepackage{glossaries} and \makeglossaries in your preamble
(after \usepackage{hyperref} if present). Then define any number of
\newglossaryentry and \newacronym glossary and acronym entries in your
preamble (recommended) or before first use in your document
proper. Finally add a \printglossaries call to locate the glossaries
list within your document structure. Then pepper your writing with
\gls{mylabel} macros (and similar) to simultaneously insert your
predefined text and build the associated glossary. File processing
must now include a call to makeglossaries followed by at least one
further invocation of latex or pdflatex -
https://en.wikibooks.org/wiki/LaTeX/Glossary
\end{comment}
\usepackage[toc, nonumberlist, nopostdot, xindy]{glossaries}
\usepackage{glossaries-extra}
\glssetcategoryattribute{general}{glossname}{title}

\makeglossaries
% Note: This package is fragile and you need to use \\ between
% paragraphs in a multi-paragraph description
%
\newglossaryentry{sensitivity}{
    name=sensitivity,
    description={When referring to \gls{differential_privacy}, the US Census
    Bureau uses the term \emph{sensitivity} to denote the impact that
    a single person can have on the computation of a statistic. In the
    \gls{differential_privacy} literature this quantity is properly referred
    to as the $l_1$ sensitivity.\\
    \\
     ``The $l_1$ sensitivity of a function $f$ captures the
    magnitude by which a single individual’s data can change the
    function $f$ in the worst case, and therefore, intuitively, the
    uncertainty in the response that we must introduce in order to
    hide the participation of a single individual.'' \parencite{dwork_algorithmic_2013} \\
    \\
    ``The sensitivity of a counting query is 1 (the
    addition or deletion of a single individual can change a count by
    at most 1)'' \parencite{dwork_algorithmic_2013} }
}

\newglossaryentry{neighboring database}{
    name=neighboring database,
    description={\Gls{differential_privacy} uses the term \textit{neighboring datasets} (for example, $D$ and $D'$) to describe two datasets that
    different in the data for one person. This might be, for example,
    a dataset of 10 people living on a block who are all 30 years old,
    and a second dataset of 10 people living on a block where 9 people
    are 30 years old and one person is 31 years old.}
}

\newglossaryentry{quality}{
    name=quality,
    description={``The degree to which a set of characteristics fulfills requirements.'' (ISO 9000)\\
    ``The totality of features and characteristics of a product or service that bear on its ability to satisfy stated or implied needs. (ISO 8402: 1986, 3.1)'' \\
    ``Quality is viewed as a multi-faceted concept. The quality
    characteristics of most importance depend on user perspectives,
    needs and priorities, which vary across groups of users. Given the
    work already done in the area of quality by several organisations,
    notably, Eurostat, IMF and Statistics Canada, the OECD was able to
    draw on their work and adapt it to the OECD. Thus quality is
    viewed in terms of seven dimensions, namely:
    \begin{itemize}
    \item relevance
    \item accuracy
    \item credibility
    \item timeliness
    \item accessibility
    \item interpretability
    \item coherence. \parencite{oecd_oecd_nodate}
    \end{itemize}
    }
}

\newglossaryentry{accuracy}{
    name=accuracy,
    description={``The closeness of computations or estimates to the
    exact or true values that the statistics were intended to
    measure.'' \parencite{oecd_oecd_nodate}}
}

\newglossaryentry{utility}{
    name=utility,
    description={TBD}
}

\newglossaryentry{LEHD}{
    name=LEHD,
    description={See Longitudinal Employer-Household Dynamics}
}

\newglossaryentry{Longitudinal Employer-Household Dynamics}{
    name=Longitudinal Employer Household Dynamics,
    description={``The Longitudinal Employer-Household Dynamics (LEHD)
    program is part of the Center for Economic Studies at the
    U.S. Census Bureau. The LEHD program produces new, cost effective,
    public-use information combining federal, state and Census Bureau
    data on employers and employees under the Local Employment
    Dynamics (LED) Partnership. State and local authorities
    increasingly need detailed local information about their economies
    to make informed decisions. The LED Partnership works to fill
    critical data gaps and provide indicators needed by state and
    local authorities.'' Source: \url{https://lehd.ces.census.gov}}
}

\newglossaryentry{anonymization}
{
    name=Anonymization,
    text=anonymization,
    description={The process of removing identifying information from data about individuals so that the identity of the data subject cannot be determined. Typically this word should be avoided and the word \textit{de-identification} used instead. This word is problematic because it is typically used to describe a process, but the word actually describes the intended outcome.}
}

\newglossaryentry{anonymized_data}
{
    name=Anonymized Data,
    description={Data that has successfully undergone an \gls{anonymization}
    process such that identities of the data subjects cannot be
    learned. Because in practice it is very difficult for
    anonymization to produce data that is truly anonymized, the
    term \textit{anonymized data} should generally be avoided.} 
}

\newglossaryentry{oxford_comma}
{
    name=Oxford Comma,
    description={The Oxford Comma is the comma that comes before the word ``and'' in a list of three or more words. Census Bureau style is \textit{not} to use an Oxford Comma. Thus phrase ``\textit{bananas, apples and oranges}'' only has one comma, and not two.}
}

\newglossaryentry{2020_census}
{
    name=2020 Census,
    description={The formal name of the decennial census of population and housing being conducted by the U.S. Census Bureau with reference to the resident population as of April 1, 2020. Note that a reference to the 2020 Census is capitalized because this is the official name of the information product. Note also that in 2000, the decennial census was named Census 2000. This usage appears in contemporaneous documents for that census; however, the current usage is 2000 Census.}
}

\newglossaryentry{census}
{
    name=Census,
    description={A census is a full enumeration of all members of a given population. In general, avoid using the word \textit{census} by itself  because the Census Bureau conducts both the decennial Census of Population and Housing and the quinquennial Economic Census and Census of Governments.}
}

\newglossaryentry{census_bureau}
{
    name=Census Bureau,
    description={Within the Executive Branch of the United States government, an agency of the Department of Commerce with the statutory name  \textit{Bureau of the Census}. In the controlled vocabulary of the Department of Commerce, the agency is designated as the \textit{U.S. Census Bureau}. See \url{https://www.commerce.gov/about/bureaus-and-offices}.  Always use \textit{U.S. Census Bureau} on first reference; \emph{Census Bureau} may be used on subsequent references. When the Department of Commerce uses \textit{Census}, capitalized and without reference to any particular year, it is using its controlled vocabulary for short agency names, for example, BEA, Census or NIST. That usage is not allowed in Census Bureau documents.}
}
 
 \newglossaryentry{decennial_census}
{
    name=Decennial Census,
    description={The constitutionally mandated Census of Population and Housing. The phrase \emph{decennial census} is acceptable on first reference. Note that a generic reference to a decennial census is not capitalized. }
}

\newglossaryentry{census_of_population_and_housing}
{
    name=Census of Population and Housing,
    description={In the United States, the full name for the ``Actual enumeration'' of the resident population specified in Article 1 of the Constitution and instantiated in 13 U.S. Code as The Census Act. Note that when referencing a generic decennial census, use the term \textit{decennial census}. When referencing a particular decennial census, use either \textit{YYYY Census} or \textit{YYYY Census of Population and Housing}, where YYYY is the date of the decennial census. The former is preferred if the context is clear. The latter is required if there is a possibility of confusion with other censuses.}
    sort=census of population and housing
}

\newglossaryentry{de-anonymization}
{
    name=De-Anonymization,
    description={This term is used by some academics to describe the process of re-identifying data that have been reportedly anonymized. Because properly anonymized data cannot be re-identified, this term should not be used.}
}

\newglossaryentry{differential_privacy}
{
    name=Differential Privacy,
    text=differential privacy,
    description={Differential privacy (DP) is a property of randomized query algorithms that controls the rate of information leakage from a database that satisfies a given schema. Differential privacy was invented by Cynthia Dwork and  in 2005 (US Patent 7698250B2, filed December 16, 2005); the first publication describing differential privacy is \textcite{dwork_calibrating_2006} \parencite[see also][]{dwork_calibrating_2016}. Differential privacy may be abbreviated \emph{DP} on second reference.}
}

\newglossaryentry{semantic_privacy}
{
    name=Semantic Privacy,
    description={Semantic privacy is a collection of properties of query algorithms that quantify the advantage of a database attacker after the release of the results of a, possibly randomized, query mechanism compared to the attacker's position prior to the release.}
}

\newglossaryentry{data}
{
    name=Data,
    description={Data are plural. For singular, use \emph{datum}.}
}

\newglossaryentry{database}
{
    name=Database,
    text=database,
    description={We adopt the nomenclature of relational database theory and structured query languages. A database is an organized, computer-readable collection of symbols, usually text or numbers, to be interpreted as an account of some enterprise or operation. A database can be updated. Updates must specify a collection of variables and their new values, which may or may not be dependent on their previous values. A database is designed to be shared among a set of users subject to the requirements controlled by a database administrator.}
}

\newglossaryentry{confidential_database}
{
    name=Confidential Database,
    text=confidential database,
    description={A database for which the administrator grants access privileges to a known, finite set of users. The users of a confidential database may not alter their access privileges, extend them to others in the user set, or add users to that set without the permission of the administrator.}
}

\newglossaryentry{confidential_statistics}
{
    name=Confidential Statistics,
    description={Statistics that are computed on the \gls{confidential_database}.}
}

\newglossaryentry{database_administrator}
{
    name=Database Administrator,
    description={The database administrator is the person or entity, designated by the database owner, who has the authority to grant and alter access privileges.}
}

\newglossaryentry{public-use_database}
{
    name=Public-Use Database,
    text=public-use database,
    description={A \gls{database} for which the administrator has granted read access privileges to any user in the world.}
}

\newglossaryentry{public-use_data}
{
    name=Public-Use Data,
    description={Synonym for \gls{public-use_database}.}
}

\newglossaryentry{table}
{
    name=Table,
    description={A two-dimensional symbol in a database in which the \glspl{record} (rows) represent entities and the variables (columns) represent attributes for which each entity may have a specific value. When a database consists of a single table, the terms database and table may be used interchangeably.}
}

\newglossaryentry{record}
{
    name=Record,
    text=record,
    description={A record is one row in a properly defined table in a database.}
}

\newglossaryentry{row}
{
    name=Row,
    description={The term \textit{row} is sometimes used a synonym for a \gls{record} in a table in database. In publications, we  use the term \textit{record} and  always indicate whether the record appears in a table of the \gls{confidential_database} or the \gls{public-use_database}. To avoid confusion, we only use the term \textit{row} when referring to a specific row of a two-dimensional matrix that appears in the referenced presentation or publication. },
    see={record}
}

\newglossaryentry{variable}
{
    name=Variable,
    description={A variable is a column in a properly defined table in a database.}
}

\newglossaryentry{column}
{
    name=Column,
    description={Column is a synonym for variable in a properly defined table in a database. The term \textit{column} may be used if the table is being represented as a two-dimensional matrix in the mathematical presentation.}
}

\newglossaryentry{geolevel}
{
    name=Geolevel,
    description={A geographical level corresponding to one of the  ``summary levels'' in the central spine of the Census Bureau Geography Division's Summary Level Chart. The current defined geolevels are nation, state, county, tract, block Group and block. A characteristic of the these Geography Division summary levels is each each one completely tiles the geographical area of the United States. That is, each point in the United States is located in a specific State, County, Tract, Block Group and Block.}
}

\newglossaryentry{geounit}
{
    name=Geounit,
    description={A specific Nation, State, county, tract, block group or block. Geounits are described by a numeric sequence consisting of the 2-digit ANSI state code, the 3-digit ANSI county code, the 6-digit census tract, the 1-digit block group, and the 4-digit block. Note that the first digit of the block \emph{is} the block group, so digits 12 and 13 are always the same digit.}
}

\newglossaryentry{attribute}
{
    name=Attribute,
    description={Attribute is a synonym for variable.}
}

\newglossaryentry{access_privileges}
{
    name=Access Privileges,
    description={The list of operations on the database that can be granted or controlled by the database administrator. We will leave this as a primitive. Access privileges include the ability to read the database and its schema. They may also include the separate ability to alter the database or its schema. The most basic access privilege is \textit{read access}.}
}

\newglossaryentry{database_schema}
{
    name=Database Schema,
    description={The mathematical description of each table in a database, defining the universe from which entities constituting the rows of the table are drawn, the allowable properties of each variable or attribute in the columns of the table, and any mandatory relations among the tables. When the database consists of a single table, the schema need not specify relations to other objects defined outside the scope of that table.}
}

\newglossaryentry{de-identification}
{
    name=De-Identification,
    description={De-identification is a ``general term for any process
    of removing the association between a set of identifying data and
    the data subject.'' (ISO/TS 25237:2008(E) Health Informatics — Pseudonymization. ISO, Geneva, Switzerland. 2008.)\\
    \\
    Removing identifiers from a dataset. Use this word instead of \textit{\gls{anonymization}} because this word describes the process, whereas \textit{anonymization} describes the desired outcome.}
}

\newglossaryentry{relation}
{
    name=Relation,
    description={A relation is a mathematical function connecting the elements of one table in a database to elements of one or more other tables in the same database. The relation is defined in the syntax of the database language that also governs the tables and the schema}
}

\newglossaryentry{relational_database}
{
    name=Relational Database,
    description={A database is relational if all of the symbols are tables, all of the tables are connected by properly defined relations among their elements, and the schema correctly defines all symbols and elements.}
}

\newglossaryentry{hierarchical_database}
{
    name=Hierarchical Database,
    description={A hierarchical database is a relational database in which all the relations can be summarized by a tree. The root table, at node $0$ of the tree, may have relations connecting it to tables at level $1$ of the tree. In general, tables at level $n$ of the tree may have relations connecting them to tables at levels $n-1$ and $n+1$. No other relations are allowed.}
}

\newglossaryentry{randomized_query}
{
    name=Randomized Query,
    description={A randomized query is a random mathematical function whose domain is a database, including its schema, and a set of legal predicates defined on the elements of that database, and whose range is the set of measurable elements consistent with the schema and the predicates. A randomized query defines a conditional distribution of observing a measurable set in the range, given the specific inputs from the domain.}
}

\newglossaryentry{randomized_query_mechanism}
{
    name=Randomized Query Mechanism,
    description={The fully specified algorithm that implements a particular randomized query.}
}

\newglossaryentry{query}
{
    name=Query,
    text=query,
    description={A query is a mathematical function whose domain is a database, including its schema, and a set of legal predicates defined on the elements of that database, and whose range is the space of values consistent with the schema and the predicates.}
}

\newglossaryentry{query_response}
{
    name=Query Response,
    text=query response,
    description={The answer to a \gls{query} when applied to a \gls{database} in its domain.}
}

\newglossaryentry{database_reconstruction}
{
    name=Database Reconstruction,
    text=database reconstruction,
    description={Given only a set of \glspl{query_response} for which the input was a particular \gls{confidential_database}, a \textit{database reconstruction} is the creation of \glspl{record} in a \gls{public-use_database} that produce exactly the same responses.}
}

\newglossaryentry{database_reconstruction_attack}
{
    name=Database Reconstruction Attack,
    text=database reconstruction attack,
    description={The use of \gls{database_reconstruction} to construct a \gls{public-use_database} for which all of the \glspl{record} or portions of all of the records must exactly match counterpart records in the \gls{confidential_database} because the records in the \gls{public-use_database} are the only ones that can produce exactly the same responses as the queries on the confidential database. The shortened form \textit{reconstruction attack} may be used if the context is clear.}
}

\newglossaryentry{approximate_database_reconstruction_attack}
{
    name=Approximate Database Reconstruction Attack,
    description={The use of \gls{database_reconstruction} to construct a set of \gls{public-use_database}s for which all of the \glspl{record}, or portions of all of the records, must be similar to counterpart records in the \gls{confidential_database} because the distance between all candidate database reconstructions in the set is small in some appropriate metric. The shortened form \textit{approximate reconstruction attack} may be used if the context is clear. See \emph{Exact Database Reconstruction Attack.}}
}

\newglossaryentry{exact_database_reconstruction_attack}
{
    name=Exact Database Reconstruction Attack,
    description={The use of \gls{database} reconstruction to construct a single \gls{public-use_database} for which all of the \glspl{record} or portions of all of the records are the unique solution that matches the queries from the \gls{confidential_database} that were the inputs to the reconstruction. The shortened form \textit{exact reconstruction attack} may be used if the context is clear. See \emph{Approximate Database Reconstruction Attack.}}
}

\newglossaryentry{database_re-identification_attack}
{
    name=Database Re-Identification Attack,
    text=database re-identification attack,
    description={An effort, whether verified or not, in which names or other 
    identifying information are attached to \gls{database} \glspl{record} that were released or reconstructed without identifying information. For example, a re-identification attack might be carried out in which detailed geographical information is assigned to public-use microdata distributed at a coarser geographic level.
    At least one of the variables in the re-identified set of records must be associated with values that determine a \gls{population_unique} among the entities that correspond to the records in the \gls{confidential_database}. The shortened form \textit{re-identification attack} may be used if the context is clear.}
}

\newglossaryentry{reconstruction-abetted_database_re-identification_attack}
{
    name=Reconstruction-Abetted Database Re-Identification Attack,
    description={A \gls{database_re-identification_attack} where the \glspl{record} of the \gls{public-use_database} were constructed in whole or in part by means of a \gls{database_reconstruction_attack}.}
}

\newglossaryentry{putative_re-identification}
{
    name=Putative Re-Identification,
    text=putative re-identification,
    description={A \gls{record} that is a member of the set of \glspl{re-identification} in the \gls{public-use_database} subjected to a \gls{database_re-identification_attack}.}
}

\newglossaryentry{putative_re-identification_rate}
{
    name=Putative Re-Identification Rate,
    text=putative re-identification rate,
    description={In a \gls{database_re-identification_attack}, the ratio of \glspl{putative_re-identification} to the total number of \glspl{record} in the \gls{confidential_database}, if known, or to an estimate of the total number of records based on public-use data from the \gls{confidential_database}.}
}

\newglossaryentry{confirmed_re-identification}
{
    name=Confirmed Re-Identification,
    description={A \gls{putative_re-identification} that correctly matches its corresponding \gls{record} in the \gls{confidential_database} on the values of the variable or variables associated with the population unique.}
}

\newglossaryentry{confirmed_re-identification_rate}
{
    name=Confirmed Re-Identification Rate,
    text=confirmed re-identification rate,
    description={In a \gls{database_re-identification_attack}, the ratio of \gls{confirmed_re-identification} to the total number of \glspl{record} in the \gls{confidential_database}. Note that no approximation is allowed here because the process of confirmation must be done on the \gls{confidential_database} itself.}
}

\newglossaryentry{population_unique}
{
    name=Population Unique,
    text=population unique,
    description={For the \glspl{record} in a \gls{database}, the values of one or more variables such that one, and only one, entity in the operation or enterprise covered by the database may have a record with those values. Note that population unique is a mathematical construct that requires specification of the universe for entities whose records may appear in the database. It does not necessarily correspond to the \textit{primary key} of any table in the database.}
}

\newglossaryentry{re-identification}
{
   name=Re-identification,
   text=re-identification,
   description={Re-identification is the process of attempting to
   discern the identities that have been removed from de-identified
   data. (NISTIR 8053)}
}

\newglossaryentry{re-identification confirmation rate}
{
    name=Re-Identification Confirmation Rate,
    text=re-identification confirmation rate,
   description={In a \gls{database_re-identification_attack}, the ratio of confirmed \glspl{re-identification} to \glspl{putative_re-identification}.}
}

\newglossaryentry{fake_data}
{
    name=Fake Data,
    description={A synonym for \gls{simulated_data}. Please don't use this term.}
}

\newglossaryentry{epsilon}
{
    name=Epsilon,
    description={The privacy-loss parameter or privacy-loss budget used by \gls{differential_privacy}. When writing about \gls{differential_privacy}, please be careful not to assume that epsilon is between 0 and 1, as it may not be.}
}

\newglossaryentry{formally_private}
{
    name=Formally Private,
    description={See formal privacy}
}

\newglossaryentry{formal_privacy}
{
    name=Formal Privacy,
    description={A collection of mathematical definitions that characterize constraints on the properties of randomized queries and the associated proofs that a collection of algorithms satisfies these properties.}
}

\newglossaryentry{privatized}
{
    name=Privatized,
    description={The output of an algorithm that possesses the property \emph{formally private}. This term is used frequently in the computer science community, but we will avoid using it because it can be confusing.}
}

\newglossaryentry{public_database}
{
    name=Public Database,
    description={A synonym for \gls{public-use_database}. This may be used if the context is clear.}
}

\newglossaryentry{public_data}
{
    name=Public Data,
    description={A synonym for \gls{public-use_database}. This may be used if the context is clear.}
}

\newglossaryentry{synthetic_database}
{
    name=Synthetic Database,
    description={A \gls{database} whose schema is identical to a particular \gls{confidential_database}, but whose \glspl{record} were constructed using a statistical model whose inputs consisted of the records in that database.}
}

\newglossaryentry{synthetic_data}
{
    name=Synthetic Data,
    description={A synonym for synthetic database. This may be used if the context is clear.}
}

\newglossaryentry{simulated_database}
{
    name=Simulated Database,
    text=simulated database,
    description={A \gls{database} whose schema is identical to a particular \gls{confidential_database}, but whose \glspl{record} were constructed using a statistical model whose inputs included only public-use data, possibly but not necessarily derived from that confidential database.}
}

\newglossaryentry{simulated_data}
{
    name=Simulated Data,
    text=simulated data,
    description={A synonym for \gls{simulated_database}. This may be used if the context is clear.}
}

\newglossaryentry{disclosure_avoidance}
{
    name=Disclosure Avoidance,
    description={The preferred term at the Census Bureau for the collection of methods known as \emph{statistical disclosure limitation} in North America and \emph{statistical disclosure control} in Europe. The Census Bureau now includes formal privacy methods in its use of the term \textit{disclosure avoidance}; however, these are not yet included in the common scientific usage of statistical disclosure limitation or control.}
}

\newglossaryentry{disclosure}
{
    name=Disclosure,
    description={The legal term for revealing the values of any variable or \gls{record}in a \gls{confidential_database} to someone else. Some disclosures are permitted. Examples include the business-related need to know that is justification for a database administrator at the Census Bureau granting a user read access to a confidential database in that person's custody. \newline \newline Some disclosures are not permitted. Examples include using authorized read access to a confidential database to view information that is not related to the project for which the access was granted (called ``browsing''), emailing unencrypted \glspl{record} from a confidential database, or publishing records or parts of records from a confidential database without the approval of the Disclosure Review Board. In general, Census Bureau technical papers and memos should not use the word \textit{disclosure} without the modifier \textit{avoidance} or \textit{limitation}. There is no such thing as a ``disclosure review.'' It is a \textit{disclosure avoidance} review. The usage within the Internal Revenue Service is different. There, both permitted and illegal disclosures are called ``disclosures.'' At the Census Bureau, the only permitted disclosures are part of the business-related need to know that permits authorized users to have read access to confidential databases. We don't call these ``disclosures,'' even though the IRS does.}
}

\newglossaryentry{statistical_disclosure_limitation}
{
    name=Statistical Disclosure Limitation,
    description={The standard scientific term in North America and in most U.S. statistical agencies for the collection of methods invented in the 1970s and refined over the next decades to protect \glspl{confidential_database} from \glspl{re-identification_attack}. The preferred term at the Census Bureau is \textit{disclosure avoidance}; however see the usage for this term.}
}

\newglossaryentry{statistical_disclosure_control}
{
    name=Statistical Disclosure Control,
    description={The standard scientific term in Europe and at some North American statistical agencies, including Statistics Canada, for the collection of methods invented in the 1970s and refined over the next decades to protect \glspl{confidential_database} from re-identification attacks. The preferred term at the Census Bureau is \textit{disclosure avoidance}.}
}

\newglossaryentry{microdata}
{
    name=microdata,
    description={``An observation data collected on an individual
    object - statistical unit.'' \parencite{oecd_oecd_nodate}\\
    \\
    Census Bureau usage is \textit{microdata}, without a hyphen.}
}

\newglossaryentry{swapping}
{
    name=Swapping,
    description={The statistical disclosure limitation technique that takes as input the \glspl{record} in a \gls{confidential_database} and makes the following manipulation of those records. Certain variables are designated as the \textit{identifiers}. Certain variables are designated as the \textit{matching variables}. The remaining variables are designated as the \textit{rest of the record}. A candidate record for swapping is selected according to a set of pre-specified conditions. Once a candidate record has been selected, a set of potential swap partner records is selected according to another set of pre-specified conditions. The values of the matching variables on the candidate record are compared to the values of the same variables on the potential swap partner records. Only the potential swap partner records that match are retained. One record from the remaining potential swap partners is selected randomly. The identifiers on the candidate record and the selected swap partner record are exchanged with some pre-specified probability. The output database contains the same number of records as the input confidential database. Unswapped records are identical to their counterparts in the input database; however, the values of the identifiers in pairs of records that were actually swapped are different from their counterparts in the input database. The output database, or selected records from it, may or may not be released as a \gls{public-use_database}.}
}

\newglossaryentry{input_noise_injection}
{
    name=Input Noise Injection,
    description={For some or all of the \glspl{record} in an input \gls{confidential_database} and for some or all of the variables in that database, the values on the corresponding record in the output database have been modified by a random function. Examples include adding random noise or flipping a binary variable by subtracting it from 1 with a pre-specified probability.}
}

\newglossaryentry{output_noise_injection}
{
    name=Output Noise Injection,
    description={A synonym for a randomized query mechanism.}
}

\newglossaryentry{generalization}
{
    name=Generalization,
    description={A synonym for \emph{coarsening} that is the conventional term in database theory. You may use either.}
}

\newglossaryentry{suppression}
{
    name=Suppression,
    description={An output database is created from an input database by deleting \glspl{record} that match a pre-specified condition and/or by mapping certain pre-specified values of variables in the input database schema to a single value in the output database schema that is defined to mean ''this value has not been copied from the input database.'' The technical verb is \textit{to suppress}.}
}

\newglossaryentry{cell_suppression}
{
    name=Cell Suppression,
    description={The input database consists of tables all of which contain values for all variables. The output database contains the same number of \glspl{record} as the input database; however, some of the values for certain variables in the output database have been suppressed. This is the common meaning of suppression in the Economic Programs Directorate.}
}

\newglossaryentry{primary_suppression_rule}
{
    name=Primary Suppression Rule,
    description={In the input database, a set of conditions on the values of one or more variables in one or more tables such that, if the value encountered on any \gls{record}for those variables meets those conditions, cell suppression is applied to the values of those variables in the output database. The primary and complementary suppression rules are designed to guard against a particular database reconstruction attack commonly called a subtraction attack.}
}

\newglossaryentry{complementary_suppression_rule}
{
    name=Complementary Suppression Rule,
    description={In the input database, a set of conditions on the values of one or more variables in one or more tables, which are derived from the conditions in the primary suppression rule, such that, if the value encountered on any \gls{record}for those variables meets those conditions, cell suppression is applied to the values of those variables in the output database. The primary and complementary suppression rules are designed to guard against a particular database reconstruction attack commonly called a subtraction attack.}
}

\newglossaryentry{model_inversion}
{
    name=Model Inversion,
    description={See \textit{training data extraction}.}
}

\newglossaryentry{subtraction_attack}
{
    name=Subtraction Attack,
    description={A \gls{database_reconstruction_attack} in which one table in the \gls{public-use_database} is subtracted from another table in the public-use database to reveal all, or a portion, of a \gls{record}in the \gls{confidential_database} that was used to produce the public-use database. A subtraction attack can result in a putative re-identification if values of variables in the public-use database for the reconstructed record are associated with a population unique.}
}

\newglossaryentry{item_suppression}
{
    name=Item Suppression,
    description={A synonym for cell suppression. Mathematically, the objects in the input database can always be expressed such that they are proper two-dimensional tables as defined in this glossary. Some publication systems at the Census Bureau, however, define the tables with a third layer, which corresponds to a particular statistic when the rows are entities and variables are features of those entities. Item suppression is cell suppression in the component two-dimensional tables of this three-dimensional representation. This is the common meaning of suppression in the LEHD program.}
}

\newglossaryentry{table_suppression}
{
    name=Table Suppression,
    description={The input database consists of tables all of which contain values for all variables. The output database contains either the same table as the input database or an empty table that corresponds to a table in the input database for which at least one cell suppression occurred.This is the common meaning of suppression in the American Community Survey.}
}



\newglossaryentry{coarsening}
{
    name=Coarsening,
    text=coarsening,
    description={Given the schema for a particular variable from an input database, the schema for the same variable in the output database defines fewer allowable values, at least one of the allowable values in the output database schema maps to two or more values from the input database schema, and no value in the input database schema maps to multiple values in the output database schema. \Gls{coarsening} is the conventional term in statistical disclosure limitation, although this style guide also allows the use of the term \gls{generalization}.}
}

\newglossaryentry{top_coding}
{
    name=Top Coding,
    description={A form of \gls{coarsening} in which all values of a particular variable in the input database schema that are greater than a pre-specified value map to that pre-specified value in the output database schema.}
}

\newglossaryentry{bottom_coding}
{
    name=Bottom Coding,
    description={A form of \emph{coarsening} in which all values of a particular variable in the input database schema that are less than a pre-specified value map to that pre-specified value in the output database schema.}
}

\newglossaryentry{formally_private_synthetic_database}
{
    name=Formally Private Synthetic Database,
    description={A \gls{database} whose schema is identical to a particular \gls{confidential_database}, but whose \glspl{record} were constructed using a model whose inputs consisted exclusively of randomized query responses that satisfied formal privacy.}
}

\newglossaryentry{formally_private_synthetic_data}
{
    name=Formally Private Synthetic Data,
    description={A synonym for \gls{formally_private_synthetic_database}. The preferred term is formally private micro-data.}
}

\newglossaryentry{than_v_then}
{
    name=Than Versus Then,
    description={\textit{Than} is a comparative conjunction. Correct usage: 10 is less than 15; I am taller than you. \textit{Then} is a coordinating conjunction. Correct usage: If you do that, then I will do this. Some grammar checkers catch this now, but not all.}
}

\newglossaryentry{i_v_me}
{
    name=I Versus Me,
    description={\textit{I} is the nominative case of the first-person singular pronoun in English. Correct usage: I want that. You and I are going out. \textit{Me} is the objective case of the first-person singular pronoun in English. Correct usage: That works for you and me. (Think: you would not say "for we," you would say "for us.") \textit{We} is the nominative case of the first-person plural pronoun in English. \textit{Us} is the objective case of the first-person plural pronoun in English. Some grammar checkers catch this now, but not all. And Norma Loquendi is not the girl next door; it's the modal usage on the Internet. But that doesn't make it correct for technical writing in English.}
}

\newglossaryentry{published_data}
{
    name=Published Data,
    description={The actual data that are published.}
}

\newglossaryentry{training_data_extraction}
{
    name=Training Data Extraction,
    description={Some kinds of machine learning systems use training data to create classifiers. Training data extraction is the process of extracting the original training data from the resulting statistical classifier. This term should be used in preference to \textit{model inversion}, because there are ways to extract data from classifiers other than model inversion. }
}

\newglossaryentry{aggregation}
{
    name=Aggregation,
    description={Aggregation is the combining of multiple things into one thing. We use the term aggregation in two ways. When speaking of geographies, we say that smaller geographies are aggregated into larger ones: census blocks are aggregated into block groups, and block groups are aggregated into census tracts. Aggregation is also used to describe the combining of multiple data \glspl{record} for the production of \emph{aggregate statistics}.
    }
}

\newglossaryentry{aggregate_statistics}
{
    name=Aggregate Statistics,
    description={Statistics that result from the processing of multiple data \glspl{record}. Prior to the introduction of \gls{differential_privacy}, it was believed that aggregate statistics were sufficient to protect individual privacy. Now we know that each publication of aggregate statistics potentially results in a small loss of privacy loss for each individual contained in the aggregate sample.}
}

\newglossaryentry{perturbation}
{
    name=Perturbation,
    description={Blank.
    }
}

\newglossaryentry{inferential_disclosure}
{
    name=Inferential Disclosure,
    description={Blank.
    }
}

\newglossaryentry{privacy_loss_budget}
{
    name=Privacy Loss Budget,
    description={Blank.
    }
}

\newglossaryentry{verification_server}
{
    name=Verification Server,
    description={A method, typically a computer server,  for secondary data analysts to assess the quality of inferences obtained with protected released data \parencite{reiter_verification_2009,barrientos_providing_2018}. The verification server sends back a signal about the quality of the inference.
    } 
}

\newglossaryentry{validation_server}
{
    name=Validation Server,
    description={A method used to verify the validity of analyses run on protected data, typically through a computer server. Validation servers send back  (protected) results from running the same analysis on the confidential data. Two examples of data produced by the Census Bureau with attached validation servers are the \href{https://www.census.gov/programs-surveys/ces/data/public-use-data/synthetic-longitudinal-business-database/validating-results.html}{SynLBD} and the \href{https://www.census.gov/programs-surveys/sipp/guidance/sipp-synthetic-beta-data-product.html}{SSB}. There is a close similarity to a \gls{remote-submission-system}.    } 
}

\newglossaryentry{remote-submission-system}
{
  name={Remote Submission System},
  description={Blank.}
}

\newglossaryentry{laplace_mechanism}
{
    name=Laplace Mechanism,
    description={Blank.
    } 
}

\newglossaryentry{geometric_mechanism}
{
    name=Geometric Mechanism,
    description={Blank.
    } 
}

\newglossaryentry{exponential_mechanism}
{
    name=Exponential Mechanism,
    description={A \gls{differential_privacy} mechanism developed by Frank McSherry and Kunal Talwar \parencite{mcsherry_mechanism_2007}.
    } 
}

\newglossaryentry{structural_zero}
{
    name=Structural Zero,
    description={Blank.
    } 
}

\newglossaryentry{top_down_approach}
{
    name=Top-Down Approach,
    description={Blank.
    } 
}

\newglossaryentry{post_processing}
{
    name=Post-Processing,
    description={Blank.
    } 
}

\newglossaryentry{Redistricting-File}
{
name={Redistricting File},
description={Blank.}
}

\newglossaryentry{DHC-P}
{
name={DHC-P},
description={Blank.}
}
\newglossaryentry{DHC-H}
{
name={DHC-H},
description={Blank.}
}
\newglossaryentry{CVAP}
{
name={CVAP},
description={Blank.}
}

\newglossaryentry{2010_demonstration_data_products}
{
    name=2010 Demonstration Data Products,
    description={The \gls{Redistricting-File}, \gls{DHC-P} and \gls{DHC-H} tables released on October 29, 2019 to assist the data user community in evaluating the \gls{TopDown} algorithm proposed for use in the 2020 Census. Details can be found at \url{https://www.census.gov/programs-surveys/decennial-census/2020-census/planning-management/2020-census-data-products/2010-demonstration-data-products.html}}
}

\newglossaryentry{TopDown}
{
 name={TopDown Algorithm},
 description={Blank.}
}

\newglossaryentry{2020_census_disclosure_avoidance_system}
{
    name=2020 Census Disclosure Avoidance System,
    description={When referring to the 2020 Disclosure Avoidance System for:
    \begin{itemize}
        \item  Redistricting File (formerly the PL94-171), DHC-P and DHC-H, or some special tabulations (e.g. \gls{CVAP}) say:
        \begin{itemize}
            \item \textit{"The \Gls{TopDown} "} NOT \textit{"\Gls{differential_privacy}"}.
        \end{itemize}
        \item Detailed DHC, AIAN Tribal Summary Data, Household-Person Joins, and any other 2020 Census tabulation not generated by the TopDown Algorithm say:
        \begin{itemize}
            \item \textit{``The proposed formal privacy algorithms for other 2020 Census data product''} NOT \textit{``\Gls{differential_privacy}''} NOR \textit{"TopDown algorithm"}.
        \end{itemize}
        \item When referring to the set of tabulations from the TopDown algorithm based on the 2010 CEF proposed for soft release of September 30, 2019, say:
        \begin{itemize}
            \item \textit{``The demonstration products using 2010 data and the TopDown algorithm''} or \textit{``the 2010 Demonstration Data Products''} NOT \textit{``Test products''} NOR \textit{``\Gls{differential_privacy} test products''} NOR other variations of the same.
        \end{itemize}
    \end{itemize}}
}

\newglossaryentry{CUI}
{
    name=CUI,
    description={See  \gls{CUI2}    },
    see={CUI2}
}
\newglossaryentry{CUI2}{
  name=Controlled Unclassified Information,
  description={``Controlled Unclassified Information (CUI) is
  information that requires safeguarding or dissemination controls
  pursuant to and consistent with applicable law, regulations, and
  government-wide policies but is not classified under Executive Order
  12526 or the Atomic Energy Act, as amended.  \\
  \\
  Executive Order 13556, ``Controlled Unclassified Information'' (the
  Order), establishes a program for managing CUI across the Executive
  branch and designates the National Archives and Records
  Administration (NARA) as Executive Agent to implement the Order and
  oversee agency actions to ensure compliance. The Archivist of the
  United States delegated these responsibilities to the Information
  Security Oversight Office (ISOO).'' \parencite{noauthor_about_2016}
  %Source: \url{https://www.archives.gov/cui/about}\\
  \\
  \\
  Controlled Unclassified Information at the US Census Bureau includes information that is protected under \gls{title13} or \gls{title26}, as well as personnel information protected under \gls{title5}. }
}

\newglossaryentry{title13}
{
name={Title 13, U.S.C.},
text=Title 13,
description={The Census Bureau is bound by Title 13 of the United States Code (13 U.S.C.A. \P 1 et seq. [2007]). These laws not only provide authority for the work we do, but also provide strong protection for the information we collect from individuals and businesses.\\
\\
Title 13 provides the following protections to individuals and businesses:
\begin{itemize}
\item     Private information is never published. It is against the law to disclose or publish any private information that identifies an individual or business such, including names, addresses (including GPS coordinates), Social Security Numbers, and telephone numbers.
\item     The Census Bureau collects information to produce statistics. Personal information cannot be used against respondents by any government agency or court.
\item   Census Bureau employees are sworn to protect confidentiality. People sworn to uphold Title 13 are legally required to maintain the confidentiality of your data. Every person with access to your data is sworn for life to protect your information and understands that the penalties for violating this law are applicable for a lifetime.
\item    Violating the law is a serious federal crime. Anyone who violates this law will face severe penalties, including a federal prison sentence of up to five years, a fine of up to \$250,000, or both.
\end{itemize} \parencite{us_code_title_1954}
}
}

\newglossaryentry{title26}
{
name={Title 26, U.S.C.} ,
text={Title 26},
description={The Internal Revenue Code (IRC) is the body of law that codifies all federal tax laws, including income, estate, gift, excise, alcohol, tobacco, and employment taxes. \\
\\
These laws constitute Title 26 of the U.S. Code (26 U.S.C.A. \P 1 et seq. [1986]) and are implemented by the Internal Revenue Service (\gls{IRS}) through its Treasury Regulations and Revenue Rulings.\\
\\
Congress made major statutory changes to Title 26 in 1939, 1954, and 1986. Because of the extensive revisions made in the Tax Reform Act of 1986, Title 26 is now known as the Internal Revenue Code of 1986 (Pub. L. No. 99-514, \P 2, 100 Stat. 2095 [Oct. 22, 1986]).\\
\\
Title 26, U.S. Code applies to the statistical work conducted by the U.S. Census Bureau's collection of IRS data about households and businesses. Title 26 provides for the conditions under which the IRS may disclose Federal Tax Returns and Return Information (\gls{FTI}) to other agencies, including the Census Bureau. Specifically, Title 26, U.S. Code 6103 (j) (1) permits the IRS to share FTI with the Census Bureau for statistical purposes in the structuring of censuses and national economic accounts, as well as for conducting related statistical activities authorized by law.\\
\\
Protection of Title 26 data\\
\\
Publication of all statistical products by the Census Bureau, including those based in whole or in part on administrative records covered by Title 26, are subject to disclosure avoidance procedures prescribed by the Census Bureau's internal Disclosure Review Board. Additionally, products using administrative records data are subject to any additional disclosure review required by the supplying agency. \parencite{us_census_bureau_title_nodate}}
}

\newglossaryentry{FTI}
{
name={Federal Tax Returns and Return Information },
description={Blank. Abbreviated as FTI.}
}

\newglossaryentry{IRS}
{
name={Internal Revenue Service},
description={Blank. Abbreviated as IRS.}
}

\newglossaryentry{title5}
{
name=Title 5,
description={Title 5 U.S.C. More details.}
}


% LocalWords:  interpretability


% Must be issued before \printglossaries
\setglossarystyle{altlisthypergroup}


%%% IF YOU HAVE ANY QUESTIONS, LOOK AT THE README.MD

\begin{document}

\title{WP22 Chapter 4: Impact of restricted access models and SDL methods on data quality and usability}

\author{% 
Anna Oganian\thanks{National Center for Health Statistics}, 
Ellen Galantucci\thanks{Bureau of Labor Statistics}, \\
Donna Miller\thanks{National Center for Health Statistics}, 
Lars Vilhuber\thanks{U.S. Census Bureau and Cornell University}, \\
Simson Garfinkel\thanks{U.S. Census Bureau}
}

\maketitle
\newpage 
\begin{center}
    This part will be removed in the final version?
\end{center}
\tableofcontents

\newpage 

\begin{abstract}
When releasing data to the public, statistical agencies and survey
organizations typically alter data values in order to protect the
confidentiality of survey respondents' identities and attribute
values.  To select among the wide variety of data alteration
methods, agencies require tools for evaluating the utility of
proposed data releases.  Such utility measures can be combined
with disclosure risk measures to gauge risk-utility tradeoffs of
competing methods.  Some examples of utility metrics are presented in this Chapter.
A  decision-theoretic formulation for evaluating
disclosure limitation procedures based on utility and risk metrics is outlined
as well. Finally, examples of utility assessments of certain families of SDL methods are given. 


\end{abstract}

\section{Introduction}\label{sec.intro}

As it was described in Chapter 3, there is a wide range of \SDL\ techniques.
These \SDL\ methods can be implemented with differing degrees of
intensity.  Generally, increasing the amount of alteration
decreases the risk of disclosure, but it also decreases the
accuracy of inferences obtainable from the released data, often
referred to as data utility \citep{Hund10}.

So,  \SDL\ practitioner needs to decide which technique, and with what
degree of intensity to use in a particular setting of data release. 
In general the approach to this problem is to employ risk-utility formulations. We assume
below that each candidate release $R$ can be characterized by a quantified
\textit{disclosure risk} $\DR (R)$ and \textit{data utility} $\DU
(R)$.  Examples of $\DU(R)$ metrics are given in Section \ref{du_metrics}.
The particular released data, $DB_{REL}$, can be selected from
the candidates in one of two ways. The first is to maximize
utility subject to an upper bound on risk, by solving an
optimization problem of the form
\begin{equation}\label{eq.opt}
\begin{array}{l}
DB_{REL} = {\arg\max}_{R \in \RS} \DU (R) \\[1ex]
\mbox{s.t. } \DR(R) \leq \alpha
\end{array}
\end{equation}
where \RS\ is the set of all possible releases. 

The second and more flexible approach is to define \textit{risk-utility
frontiers} using the partial order $\preceq_{\mathrm{RU}}$ defined
by
\begin{equation}\label{eq.rupo}
R_1 \preceq_{\mathrm{RU}} R_2 \Leftrightarrow \DR(R_2) \leq
\DR(R_1) \qquad\mbox{and}\qquad \DU(R_2) \geq \DU(R_1).
\end{equation}

When $R_1 \preceq_{\mathrm{RU}} R_2$, the $R_2$ is preferred to
$R_1$ because it has both lower disclosure risk and higher
utility. Only candidate releases on the risk-utility frontier of
maximal elements of \RS\ with respect to the partial order
(\ref{eq.rupo}) need be considered further: for any other
candidate, some element of the frontier has lower risk
\textit{and} higher utility. Calculation of the frontier can be
done using existing algorithms for finding the maxima in a set of
vectors \citep{kung-luccio-preparata75}. 


The choice among the SDL methods lying on the risk-utility
frontier lies with the data disseminator. To illustrate the first approach
described above, consider Figure \ref{fig.ruplot}, where each point
represents some SDL method characterized in terms of Utility and Risk 
measured according to certain metrics on the horizontal and vertical axes respectively.
If the risk threshold were $10\%$ (in some settings, not a very conservative value), 
then a method denoted as \NOI\ would be the preferred SDL. It is also clear from Figure \ref{fig.ruplot} that compared to \MICZ\ or \NOI\, \MICI\ produces only a minor increase
in utility at an enormous cost in terms of disclosure risk. Similarly,
\RANK\ yields only a modest improvement in disclosure
risk over \MICP\ and \NOI\, but incurs an immense
penalty in terms of data utility. Thus, in a scenario represented by Figure \ref{fig.ruplot}
 the disseminator  might prefer \NOI\ or \MICP\ .

\begin{figure}
\begin{center}
\includegraphics[width=4in]{R_U_plot.pdf}
\end{center}
\caption{Risk-Utility plot}
\label{fig.ruplot}
\end{figure}


\section{Scope of this chapter}\label{sec:scope}

We reference previous chapters' coverage of \gls{SDL} and other methods that restrict access to the full confidential data, describe various metrics that might be used to assess the utility of such methods.

\section{Access Methods: A brief listings}\label{sec:access_methods}

Access methods are one component of a disclosure avoidance system.

\section{Disclosure Avoidance Methods}\label{sec:da_methods}

To structure this chapter, we outline our framework. 

When evaluating utility, we consider the final public output product - after all output controls. Figure~\ref{fig:framework} illustrates the data flow from confidential data to output product, going through "mechanisms" that are combinations of "access control" methods ("Safe settings", "Safe people", "Safe projects", and "safe outputs") and "SDL methods" ("safe data"). (terminology from the five safes framework.

\begin{figure}
    \centering
\includegraphics[width=0.8\textwidth]{SDL+Access control.png}
    \caption{Evaluation Framework}
    \label{fig:framework}
\end{figure}

\section{Non-Data Utility Metrics}\label{sec:other_metrics}
\begin{itemize}
    \item Ease of access
    \item Timeliness of outputs
    \item others?
\end{itemize}

\section{Data Utility metrics} \label{du_metrics}
In this section we present some examples of Data Utility metrics \DU\ that can be used by the data protector to assess the quality of the masked data, which can help to choose an appropriate approach for disclosure limitation.

In the broadest sense, the utility of a particular data release is the
benefit to society of the released information. Benefits this general
are nearly impossible to quantify and measure, because they depend on
more than simply the released data. A narrower, more feasible approach
is to characterize the quality of what can be learned from the masked
data relative to what can be learned from the original data. Such
comparisons can be tailored to specific analyses or can be broadened
to global differences in distributions. Examples of both approaches are presented in the corresponding sections below.

\subsection{Generic metrics of Data Utility}

Generic metrics capture global differences between the distributions of the original and masked data. 
One class of examples of generic measures are functions of the differences between point estimates of the first and second moments (and possibly other summaries) based on the original and masked data. Another is statistical distances between the distributions of the original and masked data \cite{dfks02, gks06}, for example Kullback-Leibler divergence.  When the data are approximately multivariate normal, the \KL\ captures the differences in the distributions of the entire data,
which in turn account for differences in inferences. 
However, the \KL\ measure is not easily
interpreted when the data, or some transformed version of the
data, are not reasonably well-described by a multivariate normal
distribution, that is why it's  usage as  global metric of data utility
would be limited.



\subsubsection{Propensity score utility metric}
% AO 03/31: start editing from here next time

An example of global utility metric that can be computed for the variables of different types is a propensity score based measure(\cite{propen}).

For any binary variable $T$, the propensity
score is defined as the probability that $T=1$ given covariate values $\bx$. \cite{rr83} show
that $T$ and $\bx$ are conditionally independent given the propensity
score. Thus, when two large groups have the same distributions of
propensity scores, the groups should have similar distributions of $\bx$.

This theory suggests an approach for measuring data utility.
First, we merge (by ``stacking'') the original and masked data sets, adding a variable $T$ that equals one for all records from the masked data set and equals zero for all records from the original data set. If
variables have been dropped as part of the masking, they are also
dropped in computation of propensity scores.  Second, for each record
in the original and masked data, we compute the probability of being
in the \textit{masked} data set---the propensity
score. Third, we compare the distributions of the propensity scores in
the original and masked data.  When those distributions are similar,
the distributions of the original and masked data are similar, and so
data utility should be relatively high.

Propensity scores can be estimated via a logistic regression of the
``masked/original'' variable $T$ on functions of all variables $\bx$
in the data set.  

The similarity of the propensity scores for the masked and original
observations can be assessed in numerous ways, for example comparisons
of their percentiles in each group.  A simple summary was proposed in \cite{propen}:
\begin{equation}
U_p =  \frac{1}{N} \sum_{i=1}^{N}\left[\hat{p}_i-c\right]^2,
\end{equation}
where $N$ is the total number of records in the merged data set,
$\hat{p}_i$ is the estimated propensity score for unit $i$, and $c$
equals the proportion of units with masked data in the merged data
set. In many cases, the original and masked data sets would have the
same size $N_0$, in which case, $N = 2N_0$ and $c = 1/2$. When the
original and masked data have the same distribution, the propensity
scores for all units should approximately equal $c$, so that $U_p$ is
near zero. At the other extreme, if $\hat{p}_i$ is nearly one for
units $i$ from the masked data and nearly 0 for units from the
original data, then the two data sets are completely distinguishable
and $U_p \sim 1/4$.


This measure is sensitive to the specification of the logistic
regression used to estimate the propensity scores.  For example, using
an intercept only in the regression results in $\hat{p}_i = c$ for all
$i$, regardless of the values in the masked data.  The advice from the
literature on propensity score estimation is
useful in the data utility context as well: include all variables,
with interactions and polynomial terms, considered important to make
similar in the original and masked data.

Note, that propensity score metric is not tied to the nature of the
masking.  This allows us to compute utility values on the same scale
for any masking strategy, which facilitates comparisons of the data
quality achieved by competing strategies applied on the same data
set.

Also note, that generic metrics of data utility may be blunt in that they do not necessarily distinguish among variables. For example, an \SDL\ procedure that produces very different distributions for a subset of substantively important predictors, but matches well on the subset of substantively unimportant predictors, could be rated as higher utility than a procedure that produces the opposite effects.  Similarly, minimizing the value of the generic utility metric may not lead to optimal \DBREL\ for certain conditional distributions.  

\subsection{Analysis specific metrics}
In this section we describe several analysis-specific or so called ``narrow" measures of data utility. There are many possible ways of defining such type of metrics,  they 
are linked to the types of analyses the user would like to do on the original data.
 Obviously, we cannot cover all of them, so below we present some examples 
 that can help to illustrate the idea of such metrics.



\subsection{Confidence Interval Overlap Utility Measures}\label{subsec.ci}

Data users and analysts  are frequently interested in fitting regression models. 
This process produces not only point estimates of the coefficients, but
confidence intervals as well. Thus, it is desirable for utility measures 
to indicate when the inferences, and
not just the point estimates, from regressions using the released
data are close to the corresponding ones using the original data.

Confidence intervals are main mechanism of inference in regression
models. Therefore, one measure of utility is the degree of overlap
between confidence intervals obtained from the same regressions
fit using the \DBREL\ and \DBORIG. The greater the overlap, the
higher the utility.

\subsubsection{Interval Overlap Metric}
Consider a fixed regression on the data, with specified response
and predictors. Let $(L_{\mathrm{rel},k}, U_{\mathrm{rel},k})$ be
the lower and upper limits of the 95\% confidence interval for the
regression coefficient $\beta_{k}$ obtained from \DBREL, and let
$(L_{\mathrm{orig},k}, U_{\mathrm{orig},k})$ be the corresponding
interval obtained from \DBORIG. Let $f_{\mathrm{rel},k}$ and
$f_{\mathrm{orig},k}$ be the estimated posterior distributions of
$\beta_k$ computed under \DBREL\ and \DBORIG, respectively.  For
example, in linear regression, $f_{\mathrm{orig},k}$ is the usual
$t$-distribution on $n-p$ degrees of freedom with mean
$\hat{\beta}_{\mathrm{orig},k}$ and variance the $k$th diagonal
element in
$\hat{\sigma}^2_{\mathrm{orig}}\left(X^{'}_{\mathrm{orig}}
X_{\mathrm{orig}}\right)^{-1}$, where
$\hat{\sigma}^2_{\mathrm{orig}}$ is the estimated residual
variance obtained from fitting the regression of
$Y_{\mathrm{orig}}$ on the associated $n \times p$ matrix of
predictors, $X_{\mathrm{orig}}$, which includes a vector of ones
for the intercept.

The probability overlap in the confidence intervals for
any $\beta_{k}$ \citep{kkors06} is defined to be equal to:
\begin{equation}
I_k = \frac{1}{2}
\left[\int_{L_{\mathrm{rel},k}}^{U_{\mathrm{rel},k}}
f_{\mathrm{orig},k}(t) dt +
\int_{L_{\mathrm{orig},k}}^{U_{\mathrm{orig},k}}
f_{\mathrm{rel},k}(t) dt \right]
\end{equation}
and the interval overlap measure, \IO, as
\begin{equation}
I = \sum_{i=1}^p I_k / p
\end{equation}
where $p$ is the dimension of the predictor variable matrix, including
the intercept.

By design, $0 \leq I_k \leq 0.95$ (as is the case for $I$), with
effectively no overlap corresponding to $I_k =0$ and perfect
overlap corresponding to $I_k = 0.95$.  Averaging the two
integrals in the definition of $I_k$ helps deal with cases where
$(L_{\mathrm{orig},k}, U_{\mathrm{orig},k}) \subseteq
(L_{\mathrm{rel},k}, U_{\mathrm{rel},k})$, or vice versa.  For an
illustrative example, consider the case where
$(L_{\mathrm{orig},k}, U_{\mathrm{orig},k}) = (8, 10)$, and for
two different proposed releases the $(L_{\mathrm{rel_1},k},
U_{\mathrm{rel_1},k})=(-12, 30)$ and $(L_{\mathrm{rel_2},k},
U_{\mathrm{rel_2},k})=(3, 15)$. From a utility perspective, the
second release is clearly preferable over the first release. The
\IO\ as defined favors the second release.  A criterion that just
equals $\int_{L_{\mathrm{rel},k}}^{U_{\mathrm{rel},k}}
f_{\mathrm{orig},k}(t) dt$ does not clearly distinguish the
releases, since this integral for both procedures is essentially
one. Similar examples can be constructed to show the inadequacy of
using $\int_{L_{\mathrm{orig},k}}^{U_{\mathrm{orig},k}}
f_{\mathrm{rel},k}(t) dt$ alone.

The \IO\ does not distinguish among intervals that have $I_k$
essentially equal to zero, some of which may be ``less worse''
than others. To adjust for this, the measure can be modified by
adding some distance-based penalty when $I$ is essentially zero,
or perhaps even when $I_k$ is essentially zero for some $k$, where
distance is defined as some function of the
$|\hat{\beta}_{\mathrm{rel},k} - \hat{\beta}_{\mathrm{orig},k}|$
or of $\min\left\{|L_{\mathrm{rel},k} - U_{\mathrm{orig},k}|,
|L_{\mathrm{orig},k} - U_{\mathrm{rel},k}|\right\}$.

An alternative measure is the overlap in the interval lengths. Let
$(L_{\mathrm{over},k}, U_{\mathrm{over},k})$ be the overlap in
these intervals, defined as $\left\{b: b \geq L_{\mathrm{orig},k},
b \geq L_{\mathrm{rel},k},  b \leq U_{\mathrm{orig},k},  b \leq
U_{\mathrm{rel},k}\right\}$. Then, the average relative overlap in
the confidence intervals for any $\beta_{k}$ equals:
\begin{equation}
J_k = \frac{1}{2} \left[\frac{U_{\mathrm{over},k} -
L_{\mathrm{over},k}}{U_{\mathrm{orig},k} - L_{\mathrm{orig},k}} +
\frac{U_{\mathrm{over},k} -
L_{\mathrm{over},k}}{U_{\mathrm{rel},k} -
L_{\mathrm{rel},k}}\right].
\end{equation}
The interval overlap measure then could be defined as $J = (1/p)
\sum_{i=1}^p J_k$.

\subsubsection{Ellipsoid Overlap Metric}\label{subsec.overlap}
The \IO\ measure considers each interval separately, effectively using all
the conditional distributions of the coefficients rather than
their joint distribution. Some analysts may be interested in
simultaneous intervals, which are defined by multidimensional
ellipsoids.  So, ellipsoid overlap measure \EO\ is constructed 
based on  posterior probabilities of regions defined by ellipsoids, that is, 
Bayesian perspective is used.  Generically, let $\hat{\beta}$ be the
maximum likelihood estimate of $\beta$, the $p \times 1$ vector of
true coefficients in the regression of $Y$ on $X$, and let
$\hat{\sigma}^2$ be the estimated residual variance for that
regression. Under the standard linear regression assumptions and
assuming standard non-informative prior distributions for $\beta$
and $\sigma^2$, the $(1-\alpha)100\%$ joint highest posterior
density ellipsoid for $\beta$ is defined by all the values of
$\beta$ such that
\begin{displaymath}\label{eq_1}
\frac{(\beta
-\hat{\beta})^T(X^{T}X)(\beta-\hat{\beta})}{p\hat{\sigma}^2} \leq
F(\alpha;p, n-p)
\end{displaymath}
where $F(\alpha;p, n-p)$ is the critical value from the $F$
distribution with $p$ and $n-p$ degrees of freedom.  The ellipsoid
from the \DBORIG, which we call $E_{\mathrm{orig}}$, is obtained
by setting $\hat{\beta} = \hat{\beta}_{\mathrm{orig}}$,
$\hat{\sigma}^2 = \hat{\sigma}^2_{\mathrm{orig}}$, and $X =
X_{\mathrm{orig}}$. The ellipsoid from the \DBREL, which we call
$E_{\mathrm{rel}}$, is obtained by setting $\hat{\beta} =
\hat{\beta}_{\mathrm{rel}}$, $\hat{\sigma}^2 =
\hat{\sigma}^2_{\mathrm{rel}}$, and $X = X_{\mathrm{rel}}$.

The utility measure \EO\ is the average of two posterior
probabilities: 1) the probability of $E_{\mathrm{orig}}$ computed
using the posterior distribution of $\beta$ based on \DBREL, and
2) the probability of $E_{\mathrm{rel}}$ computed using the
posterior distribution of $\beta$ based on \DBORIG. To determine
these probabilities,  Monte Carlo simulations can be used. For the first
probability, we draw values of $\beta$ from its posterior
conditional on \DBREL\ which is a $p$-variate t-distribution with
mean $\hat{\beta}_{\mathrm{rel}}$ and covariance matrix
$\hat{\Sigma}_{\mathrm{rel}} =
\hat{\sigma}^2_{\mathrm{rel}}(X_{\mathrm{rel}}^{t}X_{\mathrm{rel}})^{-1}$
with $n-p$ degrees of freedom. We then calculate the percentage of
these drawn $\beta$ that lie within $E_{\mathrm{orig}}$.  A
similar process is used to obtain the second probability by
drawing from the posterior of $\beta$ given \DBORIG\ and finding
the percentage of these that lie inside $E_{\mathrm{rel}}$. As
with \IO, the \EO\ can be extended to any parameters whose
distribution is well-approximated by a multivariate normal
distribution.

% Here: start removing from here or drastically reducing, we may just say that 
% such and such methods have generally such and such utility or compare in such a way with other methods,
% say noise is better than mixroaggregation and microaggregation is better than swapping - or simething like that.

\subsection{On Utility Properties of some families of \SDL\ methods}
\label{subsec.simulateddata}

There is a wide plethora of SDL methods and they differ significantly in terms of utility  and risk. Another source of variability of SDL methods in terms of utility (and risk as well) is due to the fact that SDL methods can be applied with a different degree of intensity, that is, data protector can chose  different values of parameters for these methods. 
For example, for Noise addition, smaller or larger variance of Noise can be chosen, which will affect the utility and risk of the masked data. For Swapping, the percentage of swapped records can be different, for Rankswappping  the  maximal allowed difference in the ranks of the swapped records  may vary (which is a parameter of Rankswapping). For Microaggregation, a minimal number of records per group should be set up by the data protector.  There is a wide variety of SDL methods and  infinitely many method-parameter combinations each leading to a masked data set with different utility and risk. Hence, we do not intend to provide a comprehensive utility comparisons here, but instead to present some examples of generic utility characterizations of different families of SDL methods . 

One of the largest families of SDL methods (in terms of different  implementations) is a  Microaggregation family. It can   be divided into Multivariate Microaggregation   and  Univariate Microaggregation. The later group includes Microaggregation with individual ranking which consists of microaggregating each variable individually and independently from other variables and Microaggregation using projections. The later one is usually accomplished by ranking  multivariate data by projecting them onto a single axis, using either the sum of $z$-scores or the first principal component, and then aggregating data into groups of size $k$, except possibly for one group of larger size (from $k+1$ to $2k -1$).
Of all these methods, Microaggregation with individual ranking is typically the least perturbative method \citep{kkors06}. Typically, masked values obtained using this method are close to the corresponding original ones, and thus analyses performed on the masked data often lead to a very similar results to those obtained on the original data. For example, \EO\ utility metric is often high (close to $1$)  for this method \citep{kkors06}. On the other hand, the risk of re-identification using record linkage approach is high for this method as well. 
In contrast to Microaggregation using Individual Ranking, projection-based Microaggregation methods and Multivariate Microaggregation may introduce significant perturbation to the data with utility ranging from average to low according to \IO\ and \EO\ . But the re-identification risk (estimated based on record-linkage experiments) is low as well \citep{}. 

One of the desirable features of Microaggregation methods is that they  inherently satisfy requirements of k-anonymity (if this criterion for Risk  is adopted by the data protector of course). Also microaggregation methods preserve means of the original data, they preserve positivity/non-negativity  constraints in the data (if the original values are positive, so are the masked values)  which in some instances of data release  is a desirable feature. Microaggregation methods, on the other hand have a shrinking effect on the original data by reducing  the variance of the original data.
 
 $ $ 

Another large family of methods is Noise infusion, which can be implemented as additive or multiplicative noise for numerical variables.
Basic implementations of noise addition preserve  mean and the correlation structure of the original data which is an important for some statistical analyses, such as linear regression. Using data transformation Noise addition can be implemented in such a way so that not only correlation matrix  but covariance matrix is preserved as well.
 In it's basic implementation noise methods do not preserve positivity or non-negativity constraints, that is, if original variables are non-negative then masked variables are also non- negative. Multiplicative noise on the other hand can preserve positivity constraints.
 % Here: About metric-related performance of noise.
 
Standard implementations of additive and multiplicative noise does not satisfies the requirements of k-anonimity.
Noise generated from Normal distribution with specially calibrated variance, as well as noise generated from Laplace distribution with special parameters  satisfies requirements of  differential privacy.
Record linkage experiments showed that additive noise has low re-identification risk.



{\bf To be continued... A few more SDL families will be described/compared in terms of Utility}
 
\section{Donna's stuff}
ACCESS TIER: PROTECTED
• Restricted-use data behind firewall with output controlled for disclosures;
• Automated output from SDL software with use restrictions (e.g. web-based query system)
• Licensing program (user controlled infrastructure)

Access to this data is not automatic but requires one or more additional steps such as, but not limited to, a data use agreement, license, system or website account, and automated SDL tools with built-in limitations on allowable output. The number of additional requirements for access is intended to be more than the public access, but less burdensome than the restricted tier. The lowered burden should improve access, timeliness, and other aspects of data quality in a meaningful way to the data customer.

Web-Based Query Systems 
Query systems allow users to design queries to generate customized tabulations (WP 22). Also have predefined queries.

*Data stored in query systems can be protected and restricted.

Pros 
•	Run queries on data more detailed than PUFs (varies by agency—could be restricted data or perturbed data from public use files) 
o	Increases utility, improves data quality
•	Permits a wider range of analyses than does releasing only data summaries and it provides results based on actual rather than simulated microdata (Gomatam, 2005, 164)
•	Developer can build in SDL (see Gomatam, 2005, 167)

Cons 
•	No direct access to the microdata (may vary by agency)
o	Decrease utility
•	May not prevent disclosures (table splicing; can do this in CDC Wonder) 
•	Built in disclosure technique can reduce utility and quality (Gomatam, 2005, 167-168)
o	Top coding, swapping, adding noise (WP 22)
o	Prohibit key identifiers (age, race, sex) as outcome variables but permit as predictors (Gomatam, 2005, 167)
o	Disallow any transformations (Gomatam 2005, 173)
•	Functionality is limited to what the developer allows to be run in the query system
•	Expensive to implement and maintain (Haggard, 2006, 189)

New queries vs. pre-computed queries 

Examples of online query systems: 
•	NASS (Karr)
•	Australian Bureau of Statistics, TableBuilder (Chipperfield, 2019)
•	CDC’s Wonder
•	BLS 
 

ACCESS TIER: RESTRICTED ACCESS
• Licensing program (user controlled infrastructure)
• Virtual Data Enclave
• Physical Data Enclave

This is the default tier for containing data collected under a pledge of confidentiality. Critical to its modernization is to include more access options using newer technology such as commercial cloud computing and remote access/virtual enclave. This should also include viable options not constrained by the current system of Federal Statistical Research Data Centers (FSRDCs).

Virtual and Physical Data Enclaves 
There are two types of data enclaves: 1) physical data enclaves; and 2) virtual data enclaves. Both types of enclaves allow researchers to access restricted use data under highly restricted conditions which reduces disclosure risks. Some data enclaves allow researcher to access the full data sets while others require researchers to prepare data dictionaries and limit access to only the variables the researchers need to complete their analysis. 

The primary difference between the two types of data enclaves is the process by which the data are accessed. In physical data enclaves researchers must physically sit in a controlled environment at the data owner’s office or site where the data are stored. Virtual data enclaves allow researchers to access the restricted use data remotely over secure electronic lines via their personal computers while they sit in their own offices or homes. The output generated is returned to the researchers.

Both types of data enclaves increase the usability of the data. Researchers are permitted to access data not publicly available under controlled conditions. For example, [list types of data that cannot be publicly accessed: genetics data, geocoded data, detailed geography, exact dates, detailed race, income]. 

Conversely, both types of data enclaves can also decrease the utility of the data. The process for requesting access to restricted use data can be arduous and thus reduce the number of researchers who can access the data. Researchers must submit research proposals containing detailed information about the research project, the hypotheses to be tested, the data set and variables to be used in the analysis,
the empirical methods to be used, and the specific data outputs that will result from the project thus limiting exploratory analyses. Research proposals are reviewed and approved by a review committee which can take several weeks or months to complete. Additionally, users must agree to terms and conditions governing the access and use of the confidential data as well as sign nondisclosure affidavits. Some researchers are required to complete background investigations, Special Sworn Status, be citizens of the U.S.,  …etc.  Furthermore, there are costs associated with accessing restricted use data via data enclaves [include examples]. Costs reduce the utility of the data because some researchers may not have funding to complete research. Most students completing graduate or doctoral level research, living on fixed incomes, may not be able to afford to access data in enclaves. 

Both types of data enclaves allow researchers to improve the accuracy and precision of their estimates. Data available in enclaves are not subject to the statistical disclosure limitation methods that public use files are subject to prior to release. For example, detailed race/ethnicity and geography measures are typically not available on public use files due to disclosure concerns. These types of measures are available to researchers in data enclaves thus increasing the accuracy and prevision of estimates.  [Might include examples of research completed in RDCs that could not be completed using PUFs].

Data quality can also be decreased when accessing restricted use data in data enclaves. Extreme values or values representing an individual are generally removed from analysis (e.g. minima, maxima, medians). These values might be useful to researchers doing sensitivity analysis. {need to expand this section}

Licensing Program (user controlled infrastructure)

Licensing agreements permit a researcher to use restricted data offsite, but under highly restricted
conditions, as spelled out in a legally binding agreement [text from Restricted Access Procedures]. Arrangements that place restrictions on who has access, at what locations, and for what purposes access is allowed normally require written agreements between agency and users. These agreements usually subject the user to fines, being denied access in the future and/or other penalties for improper disclosure of individual information and other violations of the agreed conditions of use. Users may be subject to external audits conducted by the agency to assure terms of the agreement are being followed. Users in violation may be required to pay fines or be subject to other legal penalties [text from WP 22].

Licensing agreements require:
- a demonstrated need for sensitive data;
- authorization for all users at the requesting institution;
- signature by a senior level official and key staff;
- a data security plan;
- agreement by researchers not to identify individual research subjects or to link data received with other microdata files; and
- review of all statistical output before publication.
[text from Restricted Access Procedures].

The license is for is a specified period of time and data files must be returned or destroyed. Some licensors require fees and/or approval by an institutional review board. Additional information on this method is provided here: https://nces.ed.gov/FCSM/pdf/CDAC\_RAP.pdf

Pros for Licensing Program
•	…
•	…

Cons for Licensing Program
•	…
•	…






 

Future Placeholder for Highly Restricted Access Tier
(leave blank)



\section{Ellen's stuff - Public Use Microdata}

**Note - Ellen will edit and add to this Thursday and add citations, but I wanted to add something for now.

Many statistical agencies provide microdata files directly to the public, available online without requiring users to register or provide information about what they will do with it. 

One advantage of making public use files available is that the data can reach a much wider audience and be used much more broadly. They are generally made free for anyone who would like access, which makes them particularly appealing for students and researchers alike.

Depending on the data, they can also be extremely easy for people to use and understand. While some may require extensive codebooks and statistical knowledge to manipulate, some government agencies are moving toward online query systems that allow anyone to determine information with little effort. For example, Baltimore city has a public database of all 911 calls that can be downloaded if the user would like to. However, they also have a query system on their website that allows users to look at combinations of variables, such as locations, date and times, and types of emergencies. With a few mouse clicks, any user can create a visualization of the data that can include multiple variables. 

Providing more people easier access to microdata can allow for more queries to be run, more data to be combined, and more research to be conducted that can change public policy. Most government agencies conduct marketing research and implement marketing techniques with the goal of having more people use their data, whether tabular or microdata. By allowing access to microdata, however, there is more information that can be gathered from the data, it can be used in additional ways, and has the potential to influence more research, more policies, and more people.

In addition, wide access to public use data can often allow for those datasets to be combined with other confidential datasets that likely wouldn’t be possible if the data were only available in a restricted use setting. Most restricted use access options, such as FSRDCs or virtual data enclaves, prevent users from removing the data in any means from their location. Public use files can usually be uploaded into these settings, but because it is not possible to pull the restricted use files from their location, it is often impossible to combine two restricted use files and draw conclusions from the resulting datasets. Public use files allow data users to combine the data with confidential datasets, which can result in conclusions that would not be possible to draw without public access to the data. 

Of course, the practice of making access to data easier, especially unperturbed microdata, increases the risk of it being used for malfeasance. Even data that are believed to be anonymized can often be used to determine information about specific respondents. Examples of misused public data are numerous, from determining preferences of Netflix users  to identifying the governor of Massachusetts in medical records by pairing it with public voter identification information . The Panel of Data Access for Research Purposes proposed two recommendations to limit de-identification attacks on government data. First, all users should be notified when accessing government data that it was collected with the legal requirement to be used for only statistical purposes and users should be required to acknowledge that they have read the disclaimer before the data can be viewed. Second, government institutions should impose ``meaningful’’ penalties for users who use the data for something other than statistical purposes, such as de-identification attempts. While that recommendation was made by the panel in 2005, many government agencies have not taken these steps and still allow access to their microdata without reading a disclaimer and have no real penalties in place for misuse of the data.

While the United States has decentralized statistical agencies which largely create their own rules on data access, other countries often have tiers of access that are standard for all data programs. Stats Canada, for example, has certain datasets that are available publicly and certain datasets that can only be accessed through data research centers or remote access that is granted for researchers. However, there are also certain datasets for which Stats Canada allows researchers to access tabular data by submitting statistical programs without seeing the underlying microdata, for which tabular data is returned. This intermediate data access level can allow the program to apply disclosure methods to the tabular data but allow researchers to use the microdata, which can add some protections for the respondents.

The United Kingdom Data Center has three levels of access, as well: open, safeguarded, and controlled data. Open data is believed to be completely free of identifying data and can be used by anyone without permission or registration. Safeguarded data is believed not to have identifying information, but there could be concerns about the possibility of identifying respondents by linking the information with other dataset. To gain access to these data, users need to register and agree to certain conditions, such as not using the data to identify individuals or disseminate any identifying information. Finally, controlled data is believed to contain information that could be disclosive and is protected through a number of means, such as requiring users to register and be approved, as well as complete training to access the data.

For both Stats Canada and the UK Data Center, each dataset is evaluated for risk and the microdata are assigned a level of protection. In some cases with risky data, the data are altered in some way, such as by removing variables, to decrease the risk. This can allow different versions of the same dataset to be available at different access levels, so researcher with training and appropriate credentials can gain access to the full data while other users who may not have the necessary credentials or may not need the full dataset can use the modified data. This gives the advantage of wide access to the data while protecting the most vulnerable elements of information from the respondents.

Disclosure Protection in Public Use Files

When datasets are release for public use, they generally have some modifications made to reduce risk to respondents. That does not mean that all risk is eliminated, but most government agencies at least attempt to limit the likelihood of re-identification. At the very least, this means removing respondent identifying information, such as names, addresses, and dates of birth from the data. In many cases, this is not removed from the file entirely, but changed into some sort of aggregate variable like county or state rather than address, or year of birth rather than exact date. While this helps to protect the respondents, it can also reduce the utility of the file. If state is the only information provided regarding the location of the respondent, it prevents studies of rural versus urban respondents.

In addition, many variables in public use microdata are rounded, top-coded, or organized into certain groups. Again, these methods may reduce some amount of utility, but can protect respondents. With the availability of information available that can be merged with public use data, exact information about variables such as dollar amounts, especially for outliers, can pose a risk. However, these methods can also skew; if wages are top-coded so all earnings above a certain level are written as something lower, the averages will no longer accurately reflect the overall population.

More surveys are beginning to use some form of data perturbation to create the microdata file that is released to the public. 
The American Community Survey (ACS) at the Census Bureau, which collects information from 2.3 million housing units per year, will move to creating synthetic data and release that as their public use microdata file, rather than the actual collected data. They have begun testing methods for creating the synthetic data, but so far, only some of the properties in the original data are reflected in the synthetic data. At geographic areas lower than the state, there are even fewer properties of the original data reflected in the synthetic microdata file.

Some surveys, such as the National Longitudinal Survey of Youth at the Bureau of Labor Statistics, release public use data files with most variables include, but lacking almost all geographic information. While that helps to provide protection for respondents to the surveys, it makes it impossible to make determinations about variations across locations.




\section{Data Access - from LARS}
\input{05_02_state_of_physical_security}

%\bibliographystyle{apalike}
\bibliography{dg2-new,from_zotero}
\bibliographystyle{econ}

\section*{Glossary}
To use the glossary, use 
\begin{verbatim}
    \gls{term}  for lower-case version
    \Gls{term} for upper-case version
    \glspl{term}  for lower-case plural
    \Glspl{term} for upper-case plural version
\end{verbatim}
Example
\begin{quote}
    The use of \gls{differential_privacy} is expanding.
\end{quote}

\glsaddall
\printglossary


\end{document}
