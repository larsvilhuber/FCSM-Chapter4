In assessing the utility of a particular method, a data custodian may also want to balance additional measures in conjunction with risk and utility. Examples include

\begin{itemize}
    \item Ease of access
    \item Timeliness of outputs
    \item Eligibility breadth
    \item Cost
\end{itemize}

\paragraph{Ease of access}

Ease of access seems to be correlated with user satisfaction for a given level of utility. Public-use data (in the public domain) provide great ease of access: in general, they can simply be downloaded. As restrictions are imposed, ease of access declines. Even agreeing to a very liberal license (such as the \href{https://creativecommons.org/licenses/by/4.0/}{Creative Commons Attribution 4.0 International (CC BY 4.0)} license, which requires attribution of the source by users) via a click-through agreement may prevent machine-readability of the download mechanism. More involved user agreements, requiring user input, or even institutional agreement, further reduce the ease (and speed, see below) of user access. The typical virtual or physical enclave may have weeks if not months of controls associated with access. 

Another important aspect is the ease with which the typical user community can leverage the provided data. Even public-use data may need to be accessible, via thorough data documentation and potentially adherence to a \textbf{community data schema} (refs here). Remote computing environments - enclaves, tabulation engines, and remote-submission systems - impose additional constraints. A remote submission system may be unfamiliar to many users. Remote desktop systems may provide limited software support, and may not be compatible with some users' expectations. For instance, as of 2020, few social science students are trained in \SAS{}, and yet several remote processing systems rely on \SAS{}. Conversely, many social science students are not yet trained in Python, and some newer systems in turn rely on those. 

\paragraph{Timeliness of outputs}

When access is granted, output controls may still vary, depending on the chosen method. For instance, a licensing scheme that delegates output control to the researcher may allow for very rapid release cycles, whereas a person-mediated disclosure avoidance system with limited capacity will delay output releases by days or months in the worst case. This may make a system with lower granularity but higher release speed more attractive to some users. 


\paragraph{Eligibility}

In general, eligibility of access may be limited. Many current restricted-access systems are limited to academic researchers, or at a minimum impose that outputs be made public-use. Whereas in the United States, much data is in the public-domain, imposing no restrictions on usage, in other countries, the lowest level is often a university-centric liberal licensing system (German campus files, until recently the Canadian Data Liberation Initiative). Such systems exclude community researchers, journalists, or commercial use, unless such users collaborate with eligible users. Conversely, even among academics, various levels of access restrictions may exist. In some instances, \footnote{VERIFY - true for HRS - which is not a federal survey.} the principal investigator must be a tenure-track faculty member, potentially excluding non-tenure-track researchers and lecturers. In other countries, students may be able to get restricted-access data with greater ease than tenure-track faculty (example: Canadian RDC system).

\paragraph{Cost}

Some systems will cost more than others, whether that is through computing infrastructure costs, compliance costs (contracts and such), or in costs imposed on users (e.g., travel costs to fixed locations for access such as RDCs). Depending on whether cost sharing (or billing) can occur, this may inhibit use. Maintenance of user-side infrastructure (such as support for RDCs or infrastructure fees for secure infrastructure) may limit accessibility only to institutions with greater financial capacity. 

